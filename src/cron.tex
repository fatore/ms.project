\section{Plano de Atividades e Cronograma Previsto}\label{sec:cronograma}

Propõe-se as seguintes atividades para este projeto de pesquisa:

\begin{enumerate}
  \item \textbf{Disciplinas da pós-graduação:} Durante o ano de 2012 o candidato participou do programa Trilha Graduação-Mestrado do ICMC/USP, o qual permitiu cursar disciplinas da pós-graduação durante o último ano da graduação. O aluno obteve todos os créditos exigidos pelo programa de mestrado ao cursar seis disciplinas oferecidas pelo ICMC, sendo elas: Computação Gráfica, Visualização Computacional, Metodologia de Pesquisa em Visualização e Imagens, Introdução ao Aprendizado de Máquina, Seminários em Computação e Matemática Computacional e Preparação Pedagógica. Em todas as disciplinas o aluno obteve máximo conceito (``A'').
    
  \item \textbf{Revisão bibliográfica:} Estudo mais aprofundado dos trabalhos da área de visualização computacional. Iniciou no segundo semestre de 2012 e proporcionou a confecção deste projeto de mestrado. No primeiro semestre de 2013 deve-se manter um intenso estudo sobre as técnicas relacionadas a esta proposta. Nos semestres seguintes mantém-se uma preocupação em acompanhar os avanços da área.

  \item \textbf{Desenvolvimento e Implementação:} Desenvolvimento e implementação dos objetivos:
    \begin{enumerate}
      \item Adoção e implementação de uma metodologia para o cálculo da similaridade entre dimensões;
      \item Criação de uma representação visual que transmita as relações entre as dimensões;
      \item Coordenação entre a representação das dimensões e a dos itens;
      \item Desenvolvimento de mecanismos de seleção e combinação para redução interativa de dimensionalidade;
      \item Desenvolvimento de um mecanismo interativo para transformação do espaço de atributos;
      \item Criação de uma metáfora visual que exprima a qualidade dos resultados apresentados;
    \end{enumerate}

  \item \textbf{Avaliação de Resultados:} Avaliação dos resultados com base na comparação com métodos de redução de dimensionalidade estabelecidos na área de aprendizado de máquina;

  \item \textbf{Trabalhos Científicos:} Redação de artigos científicos, participação em congressos e escrita da tese de mestrado bem como sua apresentação para uma banca avaliadora.
\end{enumerate}

O cronograma de execução das atividades é apresentado na Tabela~\ref{table:atividades}, assumindo um projeto de duração de vinte e quatro meses.

\newcommand{\x}{\hspace*{30pt}}
\newcommand{\y}{\color{black}\rule{30pt}{7pt}}
\renewcommand{\r}{\color{gray_c}\rule{30pt}{7pt}}
\setlength{\tabcolsep}{0pt}

\begin{table}[htb] 
  \caption[Cronograma de atividades]{Cronograma de Atividades. As marcações em preto indicam atividades que são priorizadas no período.} 
  \begin{center}
    \begin{tabular}{|c|c|c|c|c|c|c|}
      \cline{2-7}
      \multicolumn{1}{l|}{} & \multicolumn{2}{c|}{2012} & \multicolumn{2}{c|}{2013} & \multicolumn{2}{c|}{2014} \\
      \hline \ Atividade\ \ 
      & 1\textordmasculine\ S. & 2\textordmasculine\ S. 
      & 1\textordmasculine\ S. & 2\textordmasculine\ S. 
      & 1\textordmasculine\ S. & 2\textordmasculine\ S. \\
      \hline \hline                                        
      %     &       2012        &       2013         &       2014       \\
      1     &\y\y    &\y\y      &\x\x     &\x\x      &\x\x     &\x\x    \\ \hline
      2     &\x\x    &\r\r      &\y\y     &\r\r      &\r\r     &\r\r    \\ \hline
   3 (a)    &\x\x    &\x\r      &\y\y     &\r\x      &\x\x     &\x\x    \\ \hline
   3 (b)    &\x\x    &\x\x      &\r\y     &\y\r      &\x\x     &\x\x    \\ \hline
   3 (c)    &\x\x    &\x\x      &\x\r     &\r\x      &\x\x     &\x\x    \\ \hline
   3 (d)    &\x\x    &\x\x      &\x\r     &\y\y      &\r\x     &\x\x    \\ \hline
   3 (e)    &\x\x    &\x\x      &\x\x     &\r\y      &\y\r     &\x\x    \\ \hline
   3 (f)    &\x\x    &\x\x      &\x\x     &\x\r      &\r\r     &\x\x    \\ \hline
      4     &\x\x    &\x\x      &\x\r     &\r\r      &\r\y     &\x\x    \\ \hline
      5     &\x\x    &\x\x      &\r\r     &\r\r      &\r\r     &\y\y    \\ \hline
      %     &       2012        &       2013         &       2014       \\
    \end{tabular}
  \end{center}
  \label{table:atividades}
\end{table}
