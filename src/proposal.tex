\section{Proposta de Projeto}

Nesta seção, a proposta deste projeto é detalhada, apresentando a motivação, objetivos, metodologia e forma de análise dos resultados.

\subsection{Motivação}

Na Seção~\ref{sec:related} apresentou-se ferramentas visuais
que buscam modificar os conjuntos de dados para torná-los
mais representativos para o problema em estudo. 

O uso de ferramentas visuais que operam sobre grandes
volumes de dados não é exclusivo aos trabalhos relacionados
ao aqui proposto. Na verdade, toda a área de
Mineração Visual de Dados~\cite{Wong1999} (MVD),
\emph{Visual Data Mining}, tem como objetivo justamente
envolver os usuários em tarefas que até em tão eram
executadas de maneira totalmente automática. A principal
motivação desta área parte do princípio de quando o usuário
consegue compreender o resultado apresentado por uma
representação visual, então ele confia neste resultado e
consegue tirar melhor proveito das análises.

Uma característica fundamental para ferramentas MVD é manter
a simplicidade em todos aspectos do sistema~\cite{Wong1999}.
No entanto, muitas das ferramentas discutidas anteriormente
se baseiam em interfaces demasiadamente complexas, as quais
exigem do usuário um certo período de treinamento para um
uso efetivo. Tendo em vista que o objetivo das ferramentas
visuais é tornar as análises mais intuitivas, qualquer tipo
de obstáculo, como a necessidade de treinamento do usuário,
pode ser desfavorável ao se comparar com os métodos
automáticos.

Um outro aspecto que deve ser levado em consideração para o
desenvolvimento dessas ferramentas é permitir seu uso em
diversos domínios~\cite{Wong1999}. Para isso, diferentes
mecanismos de interação devem ser oferecidos, já que nenhum
mecanismo será capaz de operar otimamente para todas as
aplicações. Porém, unir em um único ambiente os principais
mecanismos necessários para a modificação efetiva de
conjuntos de dados não é tarefa trivial.

Uma questão que deve ser considerada em ferramentas de
exploração de dados, sejam elas visuais ou não, trata-se de
possibilitar investigações em subconjuntos dos dados. Isto é
importante pois dificilmente o conjunto de dados apresentará
um comportamento global, sendo mais provável que existam
subconjuntos com diferentes características que devem ser
avaliadas localmente~\cite{May2011}.

Levando em consideração os aspectos mencionados nos
parágrafos acima: simplicidade da ferramenta, diversidade de
mecanismos de interação e avaliação global e local dos
dados, este projeto de mestrado se baseia no uso de
\emph{Biplots} para desenvolver uma ferramenta que apresente
essas características e ao mesmo tempo seja capaz de cobrir
as limitações dos trabalhos do atual estado da arte.

Mais especificamente o objetivo deste trabalho pode ser
declarado da seguinte maneira:

\begin{quote} \emph{``Este projeto de mestrado tem como
        objetivo desenvolver uma ferramenta visual
        interativa que permita aos usuários modificarem
        conjuntos de dados para torná-los mais
        representativos. As modificações serão realizadas a
        partir de três principais mecanismos de interação:
        seleção, extração e construção de atributos.
        Com dados melhor representados, os métodos que
        operam sobre eles, como classificadores e
        agrupadores de dados, devem apresentar melhores
    resultados. Será por meio de uma quantificação dessa
melhoria que será feita a validação da nova ferramenta
desenvolvida.''} \end{quote}

A seguir apresenta-se a metodologia proposta para alcançar
esse objetivo.

\subsection{Metodologia}

Apesar de biplot ser velho...
Acredita-se que o emprego de \emph{Biplots} proporcionará

\subsection{Forma de análise dos resultados}

Ao reduzir o número de atributos irrelevantes ou redundantes, pode-se melhorar o desempenho computacional e a precisão das técnicas operando sobre os dados, como agrupadores e classificadores de dados. Pretende-se avaliar as contribuições deste trabalho justamente pela quantificação do desempenho de tais métodos ao utilizar as técnicas desenvolvidas, seguida de uma comparação com técnicas já estabelecidas na literatura.

% O popular método K-Means será utilizado neste trabalho para se avaliar Para se avaliar o resultado obtido por agrupadores de dados, frequemente utiliza-se a medida da silheta e o índice I. A silhueta mede a .... O indice I é utilizado para...

% Falo de SVM-Precisão, Recall (FN?) e sei la o que...




