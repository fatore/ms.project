\documentclass[12pt]{article}
\usepackage[utf8]{inputenc}
\usepackage[portuges]{babel}
\usepackage{hyperref}
\usepackage[top=30mm,bottom=25mm,left=30mm,right=25mm,twoside]{geometry}

\hypersetup{
	bookmarks=true,         % show bookmarks bar?
	pdftoolbar=true,        % show Acrobat’s toolbar?
	pdfmenubar=true,        % show Acrobat’s menu?
	pdffitwindow=false,     % window fit to page when opened
	pdfstartview={FitH},    % fits the width of the page to the window
	pdftitle={Atividades Realizadas pelo Aluno},    % title
	pdfauthor={Francisco Morgani Fatore},     % author
	pdfsubject={Projeto de Iniciação Científica},   % subject of the document
	pdfnewwindow=true,      % links in new window
	colorlinks=false,       % false: boxed links; true: colored links
	linkcolor=red,          % color of internal links
	citecolor=green,        % color of links to bibliography
	filecolor=magenta,      % color of file links
	urlcolor=cyan           % color of external links
}

\title{Atividades Realizadas pelo Candidato}
\author{Francisco Morgani Fatore\\fatore@icmc.usp.br}
\date{Julho de 2012}

\begin{document}
\maketitle

\section{Formação}

\begin{itemize}

    \item 2004-2006: Ensino Profissional de nível técnico. 
        Centro Estadual de Educação Tecnológica Paula Souza,
        CEETEPS, Sao Paulo, Brasil.

    \item 2008-2012: Graduação em Ciências de Compuatação. 
        Universidade de São Paulo, USP, Sao Paulo, Brasil.

    \item 2013-atual: Mestrado em Ciências da Computação. 
        Universidade de São Paulo, USP, Sao Paulo, Brasil.

\end{itemize}

\section{Iniciação Científica}

\begin{itemize} 

    \item \textbf{Título:} ``Exploração de Dados Multidimensionais por
        meio de Árvores de Similaridade''\\ 
        \textbf{Descrição:} A proposta inicial deste
        projeto foi estudar as abordagens existentes e
        fornecer uma alternativa às projeções de dados
        multidimensionais, de modo a utilizar árvores de
        similaridade para explorar documentos e imagens.
        Porém, com o desenvolver do projeto e principalmente
        devido a uma interação com o departamento de Física
        Biológica da Universidade do Estado de São Paulo
        (Unesp) de São João do Rio Preto, realizou-se uma
        mudança no foco em que seriam aplicadas as técnicas
        estudadas. Foi deixado o foco inicial relacionado à
        análise de documentos e imagens para trabalhar com
        uma base de dados relacionada ao processo de
        Enovelamento de Proteínas, um problema fundamental
        em biofísica molecular.\\
        \textbf{Orientador:} Fernando Vieira Paulovich\\ 
        \textbf{Período:} Agosto de 2010 - Julho de 2011\\ 
        \textbf{Agência Financiadora:} Conselho
        Nacional de Desenvolvimento Científico e
        Tecnológico\\ 

 \item \textbf{Título:} ``Visualização do Processo de Enovelamento
        de Proteínas''\\ 
        \textbf{Descrição:} Este trabalho
        iniciou uma jornada para entender o funcionamento do
        processo de enovelamento de proteínas por meio do
        uso de técnicas de visualização computacional. Os
        resultados indicaram a existência de uma estrutura
        de funil, concordando com a teoria sobre
        enovelamento protéico. Esse trabalho contou com a
        colaboração de pesquisadores do departamento de
        Física Biológica da Universidade do Estado de São
        Paulo (Unesp) de São José do Rio Preto. \\
        \textbf{Orientador:} Fernando Vieira Paulovich\\ 
        \textbf{Período:} Agosto de 2011 - Julho de 2012\\ 
        \textbf{Agência Financiadora:} Conselho
        Nacional de Desenvolvimento Científico e
        Tecnológico\\ 

 \item \textbf{Título:} ``Análise de Dados Eleitorais: Uma
        abordagem baseada em visualização''\\ 
        \textbf{Descrição:} Este
        trabalho utilizou representações visuais e técnicas
        de teoria de informação para identificar padrões e
        revelar quais fatores influenciam as eleições
        brasileiras. Apontou-se características e
        peculiaridades das eleições e como os candidatos se
        relacionam no aspecto de financiamento de campanha. \\
        \textbf{Orientador:} Fernando Vieira Paulovich\\ 
        \textbf{Período:} Agosto de 2012 - Dezembro de 2012\\ 
        \textbf{Agência Financiadora: Fundação de Amparo à Pesquisa do
        Estado de São Paulo}\\ 
        
\end{itemize}


\section{Produção Bibliográfica}

\begin{itemize}

    \item MAMANI, G. H ; FATORE, F. M. ; NONATO, L. G. ;
        PAULOVICH, F. V. . User-driven Feature Space
        Transformation. Computer Graphics Forum (Print),
        2013.

\end{itemize}

\section{Participação em Eventos}

\begin{itemize}

    \item Apresentação Oral no(a) I Workshop on Visual
        Analytics Research, 2013. (Oficina)
        Dual Projections: Interactive Dimensionality
        Reduction.

    \item Apresentação de Poster / Painel no(a) 20º Siicusp,
        2012. (Simpósio)
        Visualização do Processo de Enovelamento de
        Proteínas.

    \item  Apresentação de Poster / Painel no(a) II Workshop
        da Rede de Nanobiotecnologia em Filmes Finos, 2011.
        (Oficina)
        Visualização do Funil de Enovelamento de Proteínas
        em Modelo de Rede.

    \item  Apresentação de Poster / Painel no(a) 19º
        Siicusp, 2011. (Simpósio)
        Visualização do Funil de Enovelamento de Proteínas
        em Modelo de Rede.

\end{itemize}

\end{document}
