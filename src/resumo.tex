\begin{center}
  \textbf{Resumo}
\end{center}

\begin{quotation}
\noindent
%
A exploração de conjuntos de dados multidimensionais é um
problema que envolve basicamente dois grandes desafios. O
primeiro é que raramente se conhece a dimensionalidade
intrínseca dos dados, isto é, o conjunto de variáveis
observadas que são realmente relevantes para a compreensão
do fenômeno em estudo. O segundo se relaciona com a falta de
garantias de que os fatores fundamentais que descrevem o
problema tenham sido coletados.
%
A transformação interativa de dados é uma abordagem que
utiliza técnicas de visualização computacional em busca de
resolver ou minimizar esses desafios. No entanto, os
trabalhos disponíveis na literatura apresentam limitações,
como interfaces demasiadamente complexas e mecanismos de
interação pouco flexíveis. 
%
Assim, este projeto de mestrado tem como objetivo
desenvolver em um ambiente integrado novas técnicas visuais
interativas para a transformação de dados multidimensionais.
A metodologia proposta se baseia no uso de biplots e na
ação conjunta dos mecanismos de interação para superar as
limitações do estado da arte.
%
Acredita-se que ao se utilizar as técnicas desenvolvidas
será possível tornar os conjuntos de dados mais
representativos. Logo, atividades de exploração poderão ser
realizadas com mais eficácia e retornarão melhores
resultados.
\end{quotation}

\clearpage
