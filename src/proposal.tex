\section{Proposta de Projeto}

Nesta seção, a proposta deste projeto é detalhada, apresentando a motivação, objetivos, metodologia e forma de análise dos resultados.

\subsection{Motivação}

O uso de ferramentas visuais que operam sobre grandes
volumes de dados não é exclusivo aos trabalhos relacionados
ao aqui proposto. Na verdade, toda a área de
Mineração Visual de Dados~\cite{Wong1999} (MVD),
\emph{Visual Data Mining}, tem como objetivo justamente
envolver os usuários em tarefas que até em tão eram
executadas de maneira totalmente automática. A principal
motivação desta área parte do princípio de quando o usuário
consegue compreender o resultado apresentado por uma
representação visual, então ele confia neste resultado e
consegue tirar melhor proveito das análises.

Uma característica fundamental para ferramentas MVD é manter
a simplicidade em todos aspectos do sistema~\cite{Wong1999}.
No entanto, muitas das ferramentas discutidas anteriormente
se baseiam em interfaces demasiadamente complexas, as quais
exigem do usuário um certo período de treinamento para um
uso efetivo. Tendo em vista que o objetivo das ferramentas
visuais é tornar as análises mais intuitivas, qualquer tipo
de obstáculo, como a necessidade de treinamento do usuário,
pode ser desfavorável ao se comparar com os métodos
automáticos.

Um outro aspecto que deve ser levado em consideração para o
desenvolvimento dessas ferramentas é permitir seu uso em
diversos domínios~\cite{Wong1999}. Para isso, diferentes
mecanismos de interação devem ser oferecidos, já que nenhum
mecanismo será capaz de operar otimamente para todas as
aplicações. Porém, unir em um único ambiente os principais
mecanismos necessários para a modificação efetiva de
conjuntos de dados não é tarefa trivial.

Uma questão que deve ser considerada em ferramentas de
exploração de dados, sejam elas visuais ou não, trata-se de
possibilitar investigações em subconjuntos dos dados. Isto é
importante pois dificilmente o conjunto de dados apresentará
um comportamento global, sendo mais provável que existam
subconjuntos com diferentes características que devem ser
avaliadas localmente~\cite{May2011}.

Levando em consideração os aspectos mencionados nos
parágrafos acima: simplicidade da ferramenta, diversidade de
mecanismos de interação e avaliação global e local dos
dados, este projeto de mestrado se baseia no uso de
\emph{Biplots}~\cite{Gabriel1971} para desenvolver uma
ferramenta que apresente essas características e ao mesmo
tempo seja capaz de cobrir as limitações dos trabalhos do
atual estado da arte.

Mais especificamente o objetivo deste trabalho pode ser
declarado da seguinte maneira:

\begin{quote} \emph{``Este projeto de mestrado tem como
        objetivo desenvolver uma ferramenta visual
        interativa para a transformação de dados
        multidimensionais. A metodologia proposta se baseia
        no uso de biplots e em três principais
        mecanismos de interação: seleção, extração e
        construção de atributos. Com dados melhor
        representados, os métodos que operam sobre eles,
        como classificadores e agrupadores de dados, devem
        apresentar melhores resultados. Será por meio de uma
        quantificação dessa melhoria que será feita a
validação da nova ferramenta desenvolvida.''} \end{quote}

Um biplot pode ser entendido como uma extensão dos gráficos
de dispersão (\emph{scatterplots}). Pela Figura X é possível
notar a relação entre as duas representações. Observa-se que
enquanto em um gráfico de dispersão apresenta-se apenas as
relações entre as instâncias de dados, no biplot é possível
identificar também como os atributos se relacionam entre si.

Uma outra característica de biplots que os tornam
ferramentas excepcionais é a existência de uma relação
direta entre as representações das instâncias de dados e os
atributos. Por exemplo, na Figura X é possível determinar
quais países blah.

Essas características de biplots fazem com que seu uso seja
propício para a construção de ferramentas interativas de
transformação de dados multidimensionais. No entanto,
desconhece-se qualquer trabalho na literatura que faça uso
de tal representação gráfica para a redução de
dimensionalidade ou para a construção interativa de
atributos. 

A seguir apresenta-se a metodologia proposta por este
trabalho. Descreve-se o modo de construção de biplots
e os mecanismos de interação propostos sobre tal representação.

\subsection{Metodologia}

A construção de um \emph{biplot} parte do princípio de que qualquer
matriz $S$ de tamanho $n \times m$ e posto $r$  pode
ser representada por:

\begin{equation}\label{eq:bp}
    S = XY^T
\end{equation}

onde $X$ é uma matriz $n \times r$ e $Y$ uma matriz $m
\times r$, ambas de posto $r$~\cite{Gabriel1971}. Os valores
internos da matriz $S$ são o produto escalar entre os
vetores correspondentes de $X$ e $Y$. Por exemplo, em:

\begin{equation}
    \left( \begin{array}{rrrr}
        8 &  2 &  2 & -6 \\
        5 &  0 &  3 & -4 \\
       -2 & -3 &  3 &  1 \\
            2 &  3 & -3 & -1 \\
        4 &  6 & -6 & -2\end{array}
\right) = \left( \begin{array}{rr}
         2 & 2 \\
         1 & 2 \\
        -1 & 1 \\
         1 & -1 \\
         2 & -2\end{array} 
\right) \left( \begin{array}{rrrr}
        3 &  2 &-1 & -2 \\
    1 & -1 & 2 & -1 \end{array} 
\right)
\end{equation}

o elemento $S_{2,3}$, 3, da matriz é o
produto escalar entre $X_2$ e $Y^T_3$, $1 \times \left( -1
\right) + 2 \times 2$. Em casos como o deste
exemplo, onde o posto da matriz é dois, é possível plotar os
pontos de $X$ e $Y$ no plano. Os pontos referentes a $X$ são
os \emph{pontos do biplot}, enquanto os referentes a $Y$ são
os \emph{eixos do biplot}.

Na prática, o posto de uma matriz equivale ao menor valor
entre $n$ e $m$~\cite{Greenacre2010}. Assim, ao lidar com
grandes conjuntos de dados multidimensionais esse valor será
maior que dois e consequentemente não será possível mapear
os elementos das matrizes $X$ e $Y$ no plano. Para contornar
tal situação é comum aproximar a matriz de dados original a
uma matriz de posto igual a dois e utilizar essa aproximação
para criar a representação visual.

Uma das maneiras mais adotadas para encontrar essa
aproximação é por meio da decomposição em valores
singulares, ou simplesmente SVD (\emph{Singular value
decomposition})~\cite{Kalman1996}. O uso do método SVD é
adequado para a construção de biplots, pois além de
resolver o problema da aproximação, seu resultado possui
exatamente a forma exigida pela formulação de biplots,
apresentada na Equação~\ref{eq:bp}.

Basicamente, o método SVD estabelece que qualquer matriz $Y$
de tamanho $n \times m$ e posto $r$ pode ser expressa como o
produto de três matrizes:

\begin{equation}
    Y = UD_{\alpha}V^T
\end{equation}

onde $U$ é uma matriz $n \times r$, $V$ é uma matriz $m
\times r$ e $D_\alpha$ é uma matriz diagonal $r \times r$
com números positivos $\alpha_1,\alpha_2,\ldots,\alpha_r$ em
uma ordem decrescente. 

Para se obter o formato estabelecido na Equação~\ref{eq:bp}
basta atribuir a matriz $D$ às outras matrizes. Essa
atribuição impacta no resultado visual do biplot. Ao se
atribuir $D$ a $U$ destaca-se as relações entre as
instâncias de dados. Quando isso é feito em relação a $V$
destaca-se as relações entre os atributos. E quando se
atribui parcialmente $D$ a ambas matrizes $U$ e $V$ obtém-se
um biplot simétrico que não prioriza nenhuma característica
dos dados.

Independentemente do posto da matriz ser igual a dois,
utiliza-se apenas os dois primeiros vetores de $U$ e $V$
para a criação da representação visual. Assim, a qualidade do
resultado dependerá do erro da aproximação e da
dimensionalidade intrínseca dos dados. 

Uma vez construída a representação gráfica dos dados deve-se
desenvolver os três mecanismos interativos de transformação:
seleção, extração e construção.

O mecanismo de seleção age tanto sobre os itens quanto
dimensões. Do ponto de vista de itens, seu propósito é
possibilitar que o usuário crie subconjuntos dos dados,
buscando, por exemplo, a remoção de \emph{outliers} ou a
inspeção de um grupo de interesse. 

Nas seleções sobre dimensões deseja-se solucionar
basicamente dois problemas. O primeiro, chamado de mínimo
ótimo (\emph{minimal optimal})~\cite{Kohavi1997}, consiste
em construir um subconjunto dos atributos de entrada
evitando ao máximo a redundância entre eles. Um caso prático
deste problema está na construção de um classificador,
onde ao se evitar redundância entre os atributos de entrada
pode-se fazer com que o método obtenha ganhos tanto no tempo de
execução quanto na qualidade dos resultados obtidos. Já o
segundo problema, conhecido como todos relevantes (\emph{all
relevant})~\cite{Nilsson2007}, equivale a encontrar todos os
atributos que são de algum modo relevantes para a
compreensão do fenómeno observado. Uma aplicação real deste
problema pode ser encontrada no contexto de análise
de expressões gênicas, onde deseja-se,
identificar quais genes apresentam maior relação com o
diagnóstico de alguma doença. 

O mecanismo de extração é aplicado em situações onde a
seleção não é suficiente para sintetizar as informações
contidas nos dados. Pode ser utilizado para extrair
atributos a partir de um subconjunto de atributos. No
entanto, a extração depende de métodos automáticos como PCA,
assim torna-se um mecanismo menos intuitivo ao usuário do
que a seleção. 

O mecanismo de construção interativa de atributos é o mais
complexo entre os três propostos e por isso exige um maior
detalhamento.

Como mencionado anteriormente, um aspecto importante de
ferramentas de visualização é a avaliação da incerteza dos
resultados. Como um biplot é também um mapeamento de
elementos no plano, ferramentas para avaliação deste tipo de
representação podem ser reaproveitadas. As mais populares
são \emph{stress}~\cite{Kruskal1964}, \emph{Neighborhood
Hit}~\cite{Paulovich2008} e Neighborhood
\emph{Preservation}~\cite{Paulovich2008a}. Tais medidas
podem ser transmitidas nas representações visuais por meio
do uso de cores. 

A seguir apresenta-se como os resultados obtidos serão avaliados.

\subsection{Forma de análise dos resultados}

Ao reduzir o número de atributos irrelevantes ou
redundantes, pode-se melhorar o desempenho computacional e a
precisão das técnicas operando sobre os dados.  Pretende-se
avaliar as contribuições deste trabalho justamente pela
quantificação do desempenho de tais técnicas ao utilizar as
técnicas desenvolvidas.

Normalmente as
comparações são feitas com base na taxa de acerto da
classificação com os dados originais e com os dados após a
redução de dimensionalidade~\cite{Guyon2003,Joshi2007}. No
entanto, devido a heterogeneidade dos métodos de redução
não é trivial definir uma métrica única para as
avaliações~\cite{Medeiros2011}. Assim, considera-se também
o uso de medidas de avaliação de agrupadores de dados, como
a medida da silhueta e o índice I.

Uma outra maneira que será utilizada para se avaliar a
representatividade dos conjuntos de dados é processá-los com
o intuito de classificação.
