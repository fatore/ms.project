\section{Introdução}

A exploração de conjuntos de dados é um problema abordado com frequência em computação, tanto na área acadêmica quanto na indústria~\cite{Ngai2009,Harding2006}. Tal exploração tem como objetivo uma melhor compreensão dos fenômenos que afetam a sociedade em determinados aspectos. Com base nos novos conhecimentos adquiridos, espera-se melhorar o processo de tomadas de decisões como por exemplo, previsão de condições climáticas, diagnósticos de doenças, detecções de fraude, análise de mercado, etc.

O procedimento para a exploração inicia pela coleta e armazenamento dos dados. Esta tarefa pode ser realizada por sensores, sistemas de monitoramento, simulações ou aplicações diversas que utilizam banco de dados. A meta é coletar o máximo de informação sobre um domínio e espera-se, posteriormente, extrair novos conhecimentos a partir dos dados coletados~\cite{Keim2002}. 

Uma consequência direta de se coletar o máximo de informação é que dificilmente se conhecerá a dimensionalidade intrínseca dos dados, isto é, o conjunto de variáveis observadas que realmente são relevantes para a compreensão do fenômeno observado. Isso faz com que muitas vezes se utilize todos os atributos coletados nas investigações, o que eleva o custo computacional e dificulta as análises. 

Quando o número de atributos (dimensões) utilizado nas análises for muito elevado, 50 variáveis por exemplo, há a ocorrência do problema conhecido como ``Maldição da Dimensionalidade''~\cite{Beyer1999}. Esta ``maldição'' refere-se ao fato de que algumas das propriedades geométricas de espaços bi ou tridimensionais não se mantém para espaços de maior dimensionalidade. Uma consequência prática de tal fato é que conforme o número de  dimensões aumenta, a distância entre um ponto e seu vizinho mais próximo tende a mesma distância deste ponto ao seu vizinho mais distante~\citet{Beyer1999}.

Ao enfrentar as dificuldades impostas pela alta dimensionalidade, uma abordagem adotada com frequência é o uso de métodos de redução de dimensionalidade. O objetivo desses métodos é encontrar o menor espaço dimensional que é capaz de descrever os dados mantendo informações que são relevantes segundo determinado critério. Tal abordagem é  viabilizada pelo fato de que a maioria dos conjuntos de dados possuem atributos que não são ``importantes'' para a compreensão do fenômeno observado.

A maioria dos métodos de redução de dimensionalidade são ditos métodos caixa-preta, ou seja, o usuário inspeciona apenas os dados de entrada e saída, desconhecendo o processamento interno realizado. Deste modo, esses métodos restringem a interação do usuário, o que além de tornar o processo pouco intuitivo, impede que o usuário modifique os resultados de acordo com a sua experiência na área. 

A área de visualização computacional é uma alternativa interessante em relação aos métodos automáticos para a  exploração de conjuntos de dados, pois permite que o usuário utilize sua percepção visual para detectar padrões e seu conhecimento sobre o domínio para  orientar as análises. 

%Porém, muitos dos métodos visuais são limitados pelo número de elementos e dimensões das bases de dados.

% A área \emph{Visual Analytics}~\cite{Thomas2005} surgiu como uma união entre os métodos automáticos e visualização, seu objetivo é utilizar as vantagens das duas áreas para cobrirem suas limitações individuais. Essa união tem viabilizado avanços em diversas atividades de pesquisa, inclusive na tarefa de redução de dimensionalidade. 

Este projeto de mestrado propõe o uso de projeções multidimensionais coordenadas entre itens e dimensões para apoiar o usuário na tarefa de redução de dimensionalidade de forma mais intuitiva, ágil e confiável. Além disso, propõe-se um método que permite ao usuário modificar os resultados obtidos, quando necessário, por meio da transformação do espaço de atributos.

A obtenção de conjuntos de dados concisos, ou seja, conjuntos que não contém atributos irrelevantes ou redundantes, proporcionará  uma análise mais efetiva dos dados. Por exemplo, pode se melhorar o desempenho de agrupadores e classificadores de dados. Pretende-se avaliar as contribuições deste trabalho justamente pela quantificação do desempenho de tais métodos ao utilizar as técnicas desenvolvidas, seguida de uma comparação com técnicas já estabelecidas na literatura.

% Métodos automáticos são frequentemente utilizados para compor subconjuntos de atributos que capturam a maior parte da informação contida nos dados. Porém, como o nome sugere, esses métodos fornecem poucos, ou nenhum, dispositivos de interação o que torna o processo pouco intuitivo. Técnicas de visualização computacional têm sido utilizadas com sucesso na análise exploratória de grandes conjuntos de dados, pois permitem que o usuário utilize sua percepção visual para detectar padrões presentes nos dados e seu conhecimento sobre o domínio para interagir com os dados e orientar as análises. Assim, este projeto de mestrado tem como objetivo desenvolver mecanismos de interação sobre representações visuais para apoiar o usuário na tarefa de redução de dimensionalidade de forma mais intuitiva, ágil e confiável.

% A proposta deste projeto é a elaboração de técnicas que possam ser utilizadas para analisar as relações e dependências de atributos de uma base de dados. Parte-se da hipótese de que a união entre os campos de visualização computacional e aprendizado de máquina pode viabilizar a confecção de tais técnicas.  

\clearpage
