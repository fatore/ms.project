\section{Proposta de Projeto}\label{sec:proposal}

Nesta seção, a proposta deste projeto é detalhada, apresentando a motivação, objetivos, metodologia e forma de análise dos resultados.

\subsection{Motivação}

O uso de ferramentas visuais que operam sobre grandes
volumes de dados não é exclusivo aos trabalhos relacionados
ao aqui proposto. Na verdade, toda a área de
Mineração Visual de Dados~\cite{Wong1999} (MVD),
\emph{Visual Data Mining}, tem como objetivo justamente
envolver os usuários em tarefas que até em tão eram
executadas de maneira totalmente automática. A principal
motivação desta área parte do princípio de quando o usuário
consegue compreender o resultado apresentado por uma
representação visual, então ele confia neste resultado e
consegue tirar melhor proveito das análises~\cite{Wong1999}.

Uma característica fundamental para ferramentas MVD é manter
a simplicidade em todos aspectos do sistema~\cite{Wong1999}.
No entanto, muitas das ferramentas discutidas anteriormente
se baseiam em interfaces demasiadamente complexas, as quais
exigem do usuário um certo período de treinamento para um
uso efetivo. Tendo em vista que o objetivo das ferramentas
visuais é tornar as análises mais intuitivas, qualquer tipo
de obstáculo, como a necessidade de treinamento do usuário,
pode ser desfavorável ao se comparar com os métodos
automáticos.

Um outro aspecto que deve ser levado em consideração para o
desenvolvimento dessas ferramentas é permitir seu uso em
diversos domínios~\cite{Wong1999}. Para isso, diferentes
mecanismos de interação devem ser oferecidos, já que nenhum
mecanismo será capaz de operar otimamente para todas as
aplicações. Porém, unir em um único ambiente os principais
mecanismos necessários para a modificação efetiva de
conjuntos de dados não é tarefa trivial e nenhum dos
trabalhos estudados foi capaz de realizá-la.

Uma questão que deve ser considerada em ferramentas de
exploração de dados, sejam elas visuais ou não, é
possibilitar investigações em subconjuntos dos dados. Isto é
importante pois dificilmente o conjunto de dados apresentará
um comportamento global, sendo mais provável que existam
subconjuntos com diferentes características que devem ser
avaliadas localmente~\cite{May2011}. Porém, poucos dos
trabalhos estudados atentam para esta questão.

Levando em consideração os aspectos mencionados nos
parágrafos acima: simplicidade da ferramenta, diversidade
dos mecanismos de interação e avaliação global e local dos
dados, este projeto de mestrado se baseia no uso de
\emph{Biplots}~\cite{Gabriel1971} para desenvolver uma
ferramenta que apresente essas características e que seja
capaz de superar as limitações dos trabalhos do atual estado
da arte.

De um modo geral, o objetivo deste trabalho pode ser
declarado da seguinte maneira:


Assim, este projeto de mestrado tem como objetivo
desenvolver .

\begin{quote} \emph{``Este projeto de mestrado tem como
        objetivo desenvolver em um ambiente integrado novas
        técnicas visuais interativas para a transformação de
        dados multidimensionais. A metodologia proposta se
        baseia no uso de biplots e na ação conjunta de três
        principais mecanismos de interação para superar as
        limitações do estado da arte.  Os dois primeiros,
        seleção e extração, possibilitam a redução da
        dimensionalidade dos dados em busca de eliminar
        variáveis irrelevantes e redundantes. O terceiro
        mecanismo, construção, permite que o usuário crie
        novas dimensões com base em seu conhecimento para
        representar informações ausentes nos dados.''}
\end{quote}

O estabelecimento de biplots como base para este trabalho é
adequado, pois biplots oferecem uma representação simultânea
entre itens e dimensões dos dados de forma simples. Esta
simplicidade é justificada pela semelhança entre um biplot e
um gráfico de dispersão (\emph{scatterplot}). A única
distinção entre eles é que enquanto que em um gráfico
de dispersão apresenta-se apenas as relações entre as
instâncias de dados, em um biplot é possível identificar também
como os atributos se relacionam entre si. 

Poucas técnicas visuais são capazes de apresentar
simultaneamente informações sobre itens e dimensões em uma
única representação. No entanto, mesmo entre a minoria que
apresenta essa característica, nenhuma é capaz de
estabelecer uma coerência entre as duas representações. E
essa é outra propriedade de biplots que os tornam
ferramentas únicas.

Essas características de biplots fazem com que seu uso seja
propício para a construção de ferramentas interativas de
transformação de dados multidimensionais. No entanto,
desconhece-se qualquer trabalho na literatura que faça uso
de tal representação gráfica para a redução de
dimensionalidade ou para a construção interativa de
atributos. Assim, acredita-se que mesmo os mecanismos
interativos que já foram utilizados em outros trabalhos,
como os de redução de dimensionalidade interativa,
apresentarão características únicas na metodologia aqui
proposta.

A seguir apresenta-se a metodologia proposta por este
trabalho. Descreve-se o modo de construção de biplots
e os mecanismos de interação propostos sobre tal representação.

\subsection{Metodologia}

A construção de um \emph{biplot} parte do princípio de que qualquer
matriz $S$ de tamanho $n \times m$ e posto $r$  pode
ser representada por:

\begin{equation}\label{eq:bp}
    S = XY^T
\end{equation}

onde $X$ é uma matriz $n \times r$ e $Y$ uma matriz $m
\times r$, ambas de posto $r$~\cite{Gabriel1971}. Os valores
internos da matriz $S$ são o produto escalar entre os
vetores correspondentes de $X$ e $Y$. Por exemplo, em:

\begin{equation}
    \left( \begin{array}{rrrr}
        8 &  2 &  2 & -6 \\
        5 &  0 &  3 & -4 \\
       -2 & -3 &  3 &  1 \\
            2 &  3 & -3 & -1 \\
        4 &  6 & -6 & -2\end{array}
\right) = \left( \begin{array}{rr}
         2 & 2 \\
         1 & 2 \\
        -1 & 1 \\
         1 & -1 \\
         2 & -2\end{array} 
\right) \left( \begin{array}{rrrr}
        3 &  2 &-1 & -2 \\
    1 & -1 & 2 & -1 \end{array} 
\right)
\end{equation}

o elemento $S_{2,3}$, 3, da matriz é o
produto escalar entre $X_2$ e $Y^T_3$, $1 \times \left( -1
\right) + 2 \times 2$. Em casos como o deste
exemplo, onde o posto da matriz é dois, é possível desenhar os
pontos de $X$ e $Y$ no plano. Os pontos referentes a $X$ são
os \emph{pontos do biplot}, enquanto os referentes a $Y$ são
os \emph{eixos do biplot}.

Na prática, o posto de uma matriz equivale ao menor valor
entre $n$ e $m$~\cite{Greenacre2010}. Assim, ao lidar com
grandes conjuntos de dados multidimensionais esse valor será
maior que dois e consequentemente não será possível mapear
os elementos das matrizes $X$ e $Y$ no plano. Para contornar
tal situação é comum aproximar a matriz de dados original a
uma matriz de posto igual a dois e utilizar essa aproximação
para criar a representação visual.

Uma das maneiras mais adotadas para encontrar essa
aproximação é por meio da decomposição em valores
singulares, ou simplesmente SVD (\emph{Singular value
decomposition})~\cite{Kalman1996}. O uso do método SVD é
adequado para a construção de biplots, pois além de
resolver o problema da aproximação, seu resultado possui
um formato muito similar ao exigido pela formulação de biplots,
apresentada na Equação~\ref{eq:bp}.

Basicamente, o método SVD declara que qualquer matriz $Y$
de tamanho $n \times m$ e posto $r$ pode ser expressa como o
produto de três matrizes:

\begin{equation}
    Y = UD_{\alpha}V^T
\end{equation}

onde $U$ é uma matriz $n \times r$, $V$ é uma matriz $m
\times r$ e $D_\alpha$ é uma matriz diagonal $r \times r$
com números positivos $\alpha_1,\alpha_2,\ldots,\alpha_r$ em
uma ordem decrescente. 

Para se obter o formato estabelecido na Equação~\ref{eq:bp}
basta distribuir a matriz $D$ às outras matrizes. Dependendo
do modo que essa distribuição é realizada diferentes
resultados visuais são obtidos. Ao se atribuir $D$ a $U$
destaca-se as relações entre as instâncias de dados. Quando
isso é feito em relação a $V$ destaca-se as relações entre
os atributos. E quando se atribui parcialmente $D$ a ambas
matrizes $U$ e $V$ obtém-se um biplot simétrico que não
prioriza nenhuma característica dos dados.

Independentemente do posto da matriz ser igual a dois,
utiliza-se apenas os dois primeiros vetores de $U$ e $V$
para a criação da representação visual. Assim, a qualidade
do resultado dependerá do erro da aproximação e da
dimensionalidade intrínseca dos dados. Uma vez construída a
representação gráfica dos dados deve-se desenvolver os três
mecanismos interativos de transformação: seleção, extração e
construção.

O mecanismo de seleção age tanto sobre os itens quanto
dimensões. Do ponto de vista de itens, seu propósito é
possibilitar que o usuário crie subconjuntos dos dados,
buscando, por exemplo, a remoção de \emph{outliers} ou a
inspeção de um grupo de interesse. Nas seleções sobre
dimensões pretende-se solucionar basicamente dois problemas. O
primeiro, chamado de mínimo ótimo (\emph{minimal
optimal})~\cite{Kohavi1997}, consiste em construir um
subconjunto dos atributos de entrada evitando ao máximo a
redundância entre eles. Um caso prático deste problema está
na construção de um classificador, onde ao se evitar
redundância entre os atributos de entrada pode-se fazer com
que o método obtenha ganhos tanto no tempo de execução
quanto na qualidade dos resultados obtidos. Já o segundo
problema, conhecido como todos relevantes (\emph{all
relevant})~\cite{Nilsson2007}, equivale a encontrar todos os
atributos que são de algum modo relevantes para a
compreensão do fenómeno observado. Uma aplicação real deste
problema pode ser encontrada no contexto de análise de
expressões gênicas, onde deseja-se identificar quais genes
apresentam maior relação com o diagnóstico de alguma doença. 

O mecanismo de extração parte de um subconjunto de atributos
definido pelo usuário e retorna um único atributo que busca
representar a informação contida em todo o subconjunto. Como
o processo de extração depende de métodos automáticos como
PCA, este mecanismo é menos intuitivo ao usuário do que a
seleção. No entanto, existem situações em que somente a
seleção de atributos não é suficiente para sintetizar as
informações contidas nos dados.

O objetivo do mecanismo de construção é criar novas
dimensões de dados com base em interações realizadas pelo
usuário. Essas novas dimensões têm o papel de representar
aspectos subjetivos do problema em estudo que não são
facilmente medidos ou capturados nas etapas de coleta e
armazenamento dos dados. Por se tratar de um procedimento
pioneiro na literatura, até o momento da escrita deste
projeto ainda não está claramente definido o arcabouço
matemático necessário para a construção deste mecanismo.
Cogita-se o uso das transformações inversas apresentadas por
\cite{Gladys2013} ou propriedades intrínsecas dos métodos de
decomposição, como o SVD.

Como mencionado anteriormente, um aspecto importante de
ferramentas de visualização é a avaliação da incerteza dos
resultados. Como um biplot é também um mapeamento de
elementos no plano, ferramentas voltadas para esse tipo de
representação podem ser reaproveitadas, como
\emph{stress}~\cite{Kruskal1964}, \emph{Neighborhood
Hit}~\cite{Paulovich2008} e Neighborhood
\emph{Preservation}~\cite{Paulovich2008a}. Tais medidas
podem ser transmitidas nas representações visuais por meio
do uso de cores. Assim, durante todo o processo de
investigação e transformação dos dados, o usuário está
ciente da qualidade dos resultados apresentados. 

Dentre as possíveis aplicações para a ferramenta
desenvolvida, considera-se de grande relevância as
engajadas no contexto de recuperação de informação
(\emph{Information Retrieval})~\cite{Manning2008}.
Cogita-se, inclusive, o desenvolvimento de um protótipo de um
sistema de recomendação. O sistema levaria em consideração
as interações do usuário para retornar recomendações mais
pertinentes. Acredita-se que as transformações serão de
grande valia, pois os interesses e preferências dos usuários
podem ser fatores subjetivos de difícil aquisição em etapas
de coleta de dados.

A seguir apresenta-se como será feita a análise e avaliação
dos resultados.

\subsection{Forma de Análise dos Resultados}

Com dados melhor representados, os métodos que operam sobre
eles, como classificadores e agrupadores de dados, devem
apresentar melhores resultados. Será justamente por meio de uma
quantificação dessa melhoria que será feita a validação 
deste trabalho. A referência para se definir a qualidade dos
resultados será estabelecida com base em uma comparação com
métodos automáticos.

Normalmente essas comparações são feitas com base na taxa de
acerto da classificação com os dados originais e com os
dados transformados~\cite{Guyon2003,Joshi2007}. No entanto,
devido a heterogeneidade dos métodos de redução não é
trivial definir uma métrica única para as
avaliações~\cite{Medeiros2011}. Assim, considera-se também o
uso de medidas de avaliação de agrupadores de dados, como a
medida da Silhueta~\cite{Rousseeuw1987} e o Índice
Rand~\cite{Rand1971}.
